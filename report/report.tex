% This is "sig-alternate.tex" V1.9 April 2009
% This file should be compiled with V2.4 of "sig-alternate.cls" April 2009
%
% This has been modified for use in the Principles of Computer System Design @
% DIKU, 2010/2011
\documentclass{sig-alternate}
%
% This gets rid of the copyright box which is not needed for PCSD assignments
%
\makeatletter
\def\@copyrightspace{}
\makeatother
%
% For our purposes page numbers are not so bad
%
\pagenumbering{arabic}

%
% Useful packages
%
\usepackage{url}
\usepackage[english]{babel}
\usepackage[utf8]{inputenc}
%\usepackage[british]{babel}
\usepackage{hyperref}
%\usepackage{graphicx} % uncomment if you are using graphics

\usepackage{setspace}

\newcommand{\mono}[1]{{\ttfamily#1}}

\begin{document}

\title{Advanced Algorithms - Project 1}

\numberofauthors{2} % This is individual work right???

\author{
% The command \alignauthor (no curly braces needed) should
% precede each author name, affiliation/snail-mail address and
% e-mail address. Additionally, tag each line of
% affiliation/address with \affaddr, and tag the
% e-mail address with \email.
\alignauthor
Søren Dahlgaard\\
       %\affaddr{You don't have to put an address, though you could...}\\
       \email{soerend@diku.dk}
\alignauthor
Daniel Egeberg\\
       \email{egeberg@diku.dk}
}

\maketitle

\begin{abstract}
\end{abstract}

\section{Part 1}
We assume that the vertex capacities are not used in part 1, as they are only
introduced in part 2.

\subsection{1.1}
\begin{itemize}
\item It is undirected
\item Several sources and sinks
\end{itemize}

\subsection{1.2}
\begin{itemize}
\item Introduce a super-source $s$ and edges $(s, v), v = 1..6$. Let
    $c(s,v) = \infty$ for all of these.

    Also introduce a super-sink $t$ and edges $(v, t), v = 26..29$. Let
    $c(v, t) = \infty$
\item For each edge $(u,v)$ in the graph, introduce a new vertex $v'$ and two
    new edge $(v, v')$ and $(v', u)$. Let $c(v, v') = c(v', u) = c(u, v)$.
    Also make $(u,v)$ directed.
\end{itemize}

\subsection{1.3}

\subsection{1.4}



\section{Part 2}
\subsection{2.1}
For each vertex $v$, split it into two vertices $v_{\text{in}}$ and
$v_{\text{out}}$. Replace each edge $(u, v)$ with $(u, v_{\text{in}})$
and each edge $(v, u)$ with $(v_{\text{out}}, u)$. Introduce a new edge
$(v_{\text{in}}, v_{\text{out}})$ with capacity equal to the capacity of
the intersection.

\subsection{2.2}
Introduce new constraints that the sum of in-edges must be less than or equal
to the vertex capacities. Note that this also includes the original sources and
sinks, as we have introduces super-source and super-sink.

\subsection{2.3}
Changing the graph will mean that normal max-flow algorithms apply.

Changing the LP problem is easy and a good LP library will do most of the
work for you.

\subsection{2.4}

\section{Part 3}


% That's all folks!
\end{document}
