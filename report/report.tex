% This is "sig-alternate.tex" V1.9 April 2009
% This file should be compiled with V2.4 of "sig-alternate.cls" April 2009
%
% This has been modified for use in the Principles of Computer System Design @
% DIKU, 2010/2011
\documentclass{sig-alternate}
%
% This gets rid of the copyright box which is not needed for PCSD assignments
%
\makeatletter
\def\@copyrightspace{}
\makeatother
%
% For our purposes page numbers are not so bad
%
\pagenumbering{arabic}

%
% Useful packages
%
\usepackage{url}
\usepackage[english]{babel}
\usepackage[utf8]{inputenc}
%\usepackage[british]{babel}
\usepackage{hyperref}
%\usepackage{graphicx} % uncomment if you are using graphics

\usepackage{setspace}

\newcommand{\mono}[1]{{\ttfamily#1}}

\begin{document}

\title{Advanced Algorithms - Project 1}

\numberofauthors{2} % This is individual work right???

\author{
% The command \alignauthor (no curly braces needed) should
% precede each author name, affiliation/snail-mail address and
% e-mail address. Additionally, tag each line of
% affiliation/address with \affaddr, and tag the
% e-mail address with \email.
\alignauthor
Søren Dahlgaard\\
       %\affaddr{You don't have to put an address, though you could...}\\
       \email{soerend@diku.dk}
\alignauthor
Daniel Egeberg\\
       \email{egeberg@diku.dk}
}

\maketitle

\section{Part 1}
We assume that the vertex capacities displayed in the graph are not used in
part~1, as they are only introduced in part~2.

\subsection*{Question 1.1}

$G$ is not a flow network because it is an undirected graph. Also there are
several sources (the suburbs) and sinks (the city center).

%This is not necessarily the case. The graph is simply undirected!
%, or more
%specifically because for all edges $(u,v)$ there exists an edge $(v,u)$. The
%graph also has multiple sources and sinks, which is not allowed either.

\subsection*{Question 1.2}

Fixing the problem with $G$ being undirected can be fixed by introducing a
new vertex $v'$ for each edge $(u,v)$ and new edges $(v,v')$ and $(v',u)$
such that $c(v, v') = c(v', u) = c(u, v)$. Then the graph must now be
considered directed.

The problem with multiple sources can be fixed by introducing a new
super-source $s$ and new edges $(s, v)$ with $c(s, v)=\infty$ for all $v \in
S$.

Similarly, we introduce a new super-sink $t$ and edges $(v, t)$ with
$c(v,t)=\infty$ for all $t \in T$.


\subsection*{Question 1.3}

The maximum number of cars that can enter the city per minute is 240 as seen
by running \verb+part1.py+.

\subsection*{Question 1.4}

Intuitively, one would imagine that increasing the value of any edge in the
minimum cut would increase the flow value. This is, however, not true. For
instance, increasing the value of $(22, 27)$ does not affect the maximum flow
value. The edges that give an immediate effect are $(21, 26)$, $(7, 28)$,
$(20, 27)$, $(23, 29)$ and $(24, 29)$.

This is because there are several minimum cuts, and only by increasing the
capacity of an edge common to these can we direcly increase the flow.
(Increasing the capacity of an edge unique to a minimum cut simply means that
the cut is not minimum anymore. We can still pick one of the other minimum cuts
that did not contain the edge, and it will have the same value).


\section{Part 2}
\subsection*{Question 2.1}\label{sec:2.1}
For each vertex $v$, split it into two vertices $v_{\text{in}}$ and
$v_{\text{out}}$. Replace each edge $(u, v)$ with $(u_{\text{out}},
v_{\text{in}})$. Introduce a new edge $(v_{\text{in}}, v_{\text{out}})$ with
capacity equal to the capacity of the intersection $v$.

Additionally, we connect the super-source only to the $s_{\text{in}}$s.
Likewise we only connect $t_{\text{out}}$s to the super-sink.

\subsection*{Question 2.2}
Introduce new constraints that the sum of in-edges must be less than or equal
to the vertex capacities. Note that this also includes the original sources and
sinks, as we have introduces super-source and super-sink.

\subsection*{Question 2.3}
Changing the graph will mean that normal max-flow algorithms apply. These are
faster than LP algorithms, so one should be able to get an answer more
efficiently.

Changing the LP problem is a lot easier and a good LP library will do most of
the work for you.

\subsection*{Question 2.4}

Introducing these new constraints lowers the maximum flow to 210 as seen by
running \verb+part2.py+.

\section{Part 3}

\subsection*{Question 3.1}\label{sec:3.1}


Keeping the same definition of capacity contraint, flow conservation and
flow value, we introduce a new constraint: \textit{vertex capacity constraint}:

\[
\sum_{v\in V} f(u, v) \le c(u) \ge \sum_{v\in V} f(v, u)
\]

where $c(v)$ is the vertex capacity of $v$. This has to hold for all $u\in V$.

\subsection*{Question 3.2}

We define the flow $f' : V \times V \to \mathbb{R}$ as:
\begin{align}
    \label{eq1} f'(u_{\text{out}}, v_{\text{in}}) &= f(u,v) \\
    \label{eq2} f'(v_{\text{in}}, v_{\text{out}}) &= \max\left(\sum_{u\in V} f(u, v), \sum_{u\in V} f(v, u)\right) \\
    \label{eq3} f'(u,v) &= 0 \qquad\qquad\text{otherwise.}
\end{align}

Here $u_{\text{out}}$ and $u_{\text{in}}$ refer to the in- and out-vertex of
$u$ created by the transformation described in \autoref{sec:2.1}.

The $\max$ term in (\ref{eq2}) to make sure we get the right flow between the
in- and out-vertex of the source and the sink respectively.

\subsection*{Question 3.3}

To show that $f'$ is a flow we need to show that the flow conservation and
capacity constraint defined in the book holds. Let's start with the flow
conservation. We have that the following holds for all
$u\in V- \{s\}$:

\begin{align}
    \label{eq4} \sum_{v\in V} f(v,u) &= \sum_{v_{\text{out}} \in V'} f(v_{\text{out}}, u_{\text{in}}) \\
    \label{eq5} &= \sum_{v \in V'} f'(v, u_{\text{in}}) \\
    \label{eq6} &= f'(u_{\text{in}}, u_{\text{out}}) \\
    \label{eq7} &= \sum_{v \in V'} f'(u_{\text{in}}, v)
\end{align}

Here (\ref{eq4}) follows from the definition of flow in (\ref{eq1}).
(\ref{eq5}) follows from (\ref{eq3}) since all the edges from a vertex
$v$ to $u_{\text{in}}$ not in (\ref{eq4}) have a flow of $0$. (\ref{eq6})
follows directly from the definition in (\ref{eq2}) and the fact that flow
conservation holds for the original flow $f$. There is a special case for the
in-vertex of the sink, but we know that the flow going into the sink is bigger
than the flow going out, so (\ref{eq6}) also holds in this case. (\ref{eq7})
follows from (\ref{eq3}). That is, the only edge coming out of $u_{\text{in}}$
is the one to $u_{\text{out}}$.

This shows that the flow coming into an \textit{in-vertex} is equal to the
flow going out. Note that the super-source and super-sink do not have an in-
or out-vertix.

A symmetric proof shows the same for out-vertices. Note that since all vertices
except for the super-source and super-sink are in- or out-vertices, this shows
that flow conservation holds for $f'$.

In order to show the capacity constraint we need to consider the three cases
in \autoref{eq1} to \ref{eq3}.

\begin{equation}
    \label{eq8} f(u,v) \leq c(u,v) \Rightarrow f'(u_{\text{out}}, v_{\text{in}}) \leq c'(v_{\text{out}}, v_{\text{in}})
\end{equation}

This follows directly from the definition of $f'$ and $c'$. For the second
case we have:

\begin{align}
    \label{eq9} f'(v_{\text{in}}, v_{\text{out}}) &= \max\left(\sum_{u\in V} f(u, v), \sum_{u\in V} f(v, u)\right) \\
    \label{eq10}    &\leq c(v) \\
    \label{eq11}    &= c'(v_{\text{in}}, v_{\text{out}})
\end{align}

(\ref{eq9}) follows from (\ref{eq2}). (\ref{eq10}) follows from the
vertex capacity constraint defined in (\ref{sec:3.1}).
(\ref{eq11}) follows from the definition of capacity in $G'$.

For the last case the flow is $0$, so the capacity constraint holds trivially.

\subsection*{Question 3.4}

Let us start by looking at the definition of the flow $|f|$ and $|f'|$.
\begin{equation}
    |f| = \sum_{v \in V} f(s,v) - \sum_{v \in V} f(v,s)
\end{equation}

\begin{equation}
    |f'| = \sum_{v \in V'} f'(s_{\text{in}},v) - \sum_{v \in V'} f'(v,s_{\text{in}})
\end{equation}

We want to show that the two are equal.

\begin{align}
    \label{eq14} \sum_{v \in V} f(s, v) &= f'(s_{\text{in}}, s_{\text{out}}) \\
    \label{eq15}    &= \sum_{v \in V'} f'(s_{\text{in}}, v)
\end{align}

(\ref{eq14}) follows from (\ref{eq2}) because we know that the flow leaving
the source is greater than or equal to the flow coming into the source.
(\ref{eq15}) is because the edge in (\ref{eq14}) is the only one leaving
$s_{\text{in}}$.

\begin{align}
    \label{eq16} \sum_{v \in V} f(v, s) &= \sum_{v_{\text{out}} \in V'} f'(v_{\text{out}}, s_{\text{in}}) \\
    \label{eq17}    &= \sum_{v \in V'} f'(v, s_{\text{in}})
\end{align}

This shows that the two flows are equal.\qed


% That's all folks!
\end{document}
