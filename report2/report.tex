% This is "sig-alternate.tex" V1.9 April 2009
% This file should be compiled with V2.4 of "sig-alternate.cls" April 2009
%
% This has been modified for use in the Principles of Computer System Design @
% DIKU, 2010/2011
\documentclass{sig-alternate}
%
% This gets rid of the copyright box which is not needed for PCSD assignments
%
\makeatletter
\def\@copyrightspace{}
\makeatother
%
% For our purposes page numbers are not so bad
%
\pagenumbering{arabic}

%
% Useful packages
%
\usepackage{url}
\usepackage[english]{babel}
\usepackage[utf8]{inputenc}
%\usepackage[british]{babel}
\usepackage{hyperref}
%\usepackage{graphicx} % uncomment if you are using graphics

\usepackage{setspace}
\usepackage{tikz}
\usepackage{clrscode3e}

\newcommand{\mono}[1]{{\ttfamily#1}}

\begin{document}

\title{Advanced Algorithms - Project 2}

\numberofauthors{2} % This is individual work right???

\author{
% The command \alignauthor (no curly braces needed) should
% precede each author name, affiliation/snail-mail address and
% e-mail address. Additionally, tag each line of
% affiliation/address with \affaddr, and tag the
% e-mail address with \email.
\alignauthor
Søren Dahlgaard\\
       %\affaddr{You don't have to put an address, though you could...}\\
       \email{soerend@diku.dk}
\alignauthor
Daniel Egeberg\\
       \email{egeberg@diku.dk}
}

\maketitle

\section*{Part 1}
\subsection*{Question 1.1.a}

Given a graph $G=(V,E)$ and a number $d \in \mathbb{R}^+$, find a shortest
tour of all subsets $V'\subseteq V$ such that

\begin{equation}
    \label{eqn:tcp}
    \forall v \in V : \exists v' \in V' : \text{dist}(v,v') \leq d.
\end{equation}

The related decision problem is that given a graph $G=(V,E)$ and a number $d
\in \mathbb{R}^+$, does a tour exist of at most length $k$ such that
\autoref{eqn:tcp} is satisfied.

The corresponding language, $TCP$, is defined as follows:

\begin{align*}
    TCP = \{\langle G, c, d, k \rangle\} :& G = (V,E) \text{ is a complete graph,} \\
        & c : V\times V\to \mathbb{R}^+ \\
        & d \in \mathbb{R}^+, \\
        & k \in \mathbb{R} \text{ , $G$ has a TCP tour with cost at} \\
        & \qquad \text{most $k$.}
\end{align*}


\subsection*{Question 1.1.b}
To show that $TCP\in NP$ we can use the following algorithm to verify a
certificate. Let $T$ be the vertices on the tour in order:

\begin{codebox}
\Procname{$\proc{Verify-TCP}(G, c, d, k, T)$}
\li \For $v\in V$
\li \Do
        Check that there is a $u\in T$ so $c(v, u) \le d$.
\li     If not \Return $\const{False}$
    \End
\li $len \gets 0$
\li \For $i \gets 1$ \To $\attrib{T}{size} - 1$
\li \Do
        $len \gets len + c(T[i], T[i+1])$
    \End
\li \If $len + c(\attrib{T}{first}, \attrib{T}{last}) \ge k$
\li \Then
        \Return $\const{False}$
\li \Else
        \Return $\const{True}$
    \End
\end{codebox}

To show that $TCP$ is NP-hard we show that $HAM-CYCLE \le_p TCP$.
To transform any hamilton cycle problem into a TCP problem do the following
steps:

\begin{itemize}
\item Construct a graph $G' = (V, E')$, where
    $E' = \{(u,v) : u,v\in V, u\ne v\}$.
\item Set $c'(u,v) = 0$ if $(i,j)\in E$ and $1$ otherwise.
\item Set $d=0$
\item Set $k=0$
\end{itemize}

It is clear that $G'$ is a connected graph and that $c'$ is a cost function
for this graph. This transformation can easily be done in polynomial time.

We will now show that $G$ is hamiltonian $\Leftrightarrow$ $G'$ has a TCP tour
of length at most $0$.

\begin{proof}
$\Rightarrow$: If $G$ is hamiltonian, there is a hamilton cycle that visits
every vertex in $V$ using edges in $E$. Recall that these edges have weight $0$
in $E'$. Since the vertices in $G'$ is the same as in $G$, visiting the
vertices in the same order in $G'$ will yield a tour of length $0$. Because it
visits every vertex it must be a TCP tour.

$\Leftarrow$: If there is a TCP tour of $G'$ of length $0$ it must visit every
vertex $V$ because the viewing distance $d=0$. Also we have
$c(u,v) \ge 0, \forall u,v\in V$. So if the tour has total cost $0$ each edge
in the tour must have cost $0$. The only edges that have cost $0$ are those
also in $E$. Therefore the tour must also be a hamiltonian cycle in $G$.
\end{proof}

\subsection*{Question 1.2}
See \autoref{fig:tcp}. In this case the optimal tour has length $0$ ($a$ =>
$a$), but a 1-tree obviously has much bigger length.

The big blue circle is the viewing distance from the center vertex.

\tikzstyle{vertex}=[circle,fill=black!25,minimum size=20pt,inner sep=0pt]
\begin{figure}
\begin{center}
\begin{tikzpicture}
    \fill[fill=blue!10!green!10!,draw=blue,dotted,thick] (0,0) circle (2);
    \node[vertex] (a) at (0,0) {$a$};
    \node[vertex] (b) at (0,1) {$b$};
    \node[vertex] (c) at (1,0) {$c$};
    \node[vertex] (d) at (-1,0) {$d$};
    \node[vertex] (e) at (0,-1) {$e$};
\end{tikzpicture}
\end{center}
\caption{Example of a graph where the smallest 1-tree is worse than the
    optimal TCP tour}
\label{fig:tcp}
\end{figure}

\subsection*{Question 1.3}
Inspired by the ILP for the traveling salesman problem we use a decision
variable for each edge.

We wish to minimize the length of the edges in the path, ie:

\begin{equation}
    \text{minimize } \sum_{i=0}^{n}\sum_{j=0}^{n} d_{ij}x_{ij}
\end{equation}

In order to specify the constraints we introduce a notation: Let $v(S) =
\{v\in V : \exists a\in S, d_{av} \le d\}$. If $S = i$ for some we simply
write $v(i)$. Using this notation we get the following constraints:

\begin{equation}
    \label{eqn:vicinity}
    \sum_{j \in v(i)} \sum_{k = 0}^{n} x_{kj} \geq 1
    \qquad i\in \{1, 2, \ldots, n\}
\end{equation}

\begin{equation}
    \label{eqn:degree}
    \sum_{j=0}^{n} x_{ij} + \sum_{j=0}^{n} x_{ji} \le 2
    \qquad i \in \{1,2,\ldots,n\} \\
\end{equation}
\begin{equation}
    \label{eqn:degree2}
    \sum_{j=0}^{n} x_{ij} - \sum_{j=0}^{n} x_{ji} = 0
    \qquad i \in \{1,2,\ldots,n\} \\
\end{equation}

\begin{equation}
    \label{eqn:subtour}
    \sum_{i \in S} \sum_{j \in V\setminus S} x_{ij} \ge 2
    \qquad S\subset V, v(S) \ne V, v(V\setminus S) \ne V
\end{equation}

\begin{equation}
    x_{ij} \in \{0,1\}
    \qquad (i,j) \in E
\end{equation}

Where (\ref{eqn:vicinity}) makes sure that each node can be seen from the
path. (\ref{eqn:degree}) and (\ref{eqn:degree2}) makes sure that each node has
exactly $2$ or $0$ incident edges in the path.

(\ref{eqn:subtour}) is a modified version of the subtour constraint. It says
that for any cut $(S, V\setminus S)$ if neither of these sets can ``see'' the
whole graph, there has to be at least two edges crossing this cut. Below
follows a proof that this constraint is sufficient and necessary, said in
other words:

\[
    \sum_{i\in S} \sum_{j\in V\setminus S} x_{ij} \ge 2 \Leftrightarrow
    \begin{matrix} \text{ No subtours not ``seeing''} \\
                   \text{ the whole graph in $G$} \end{matrix}
\]

\begin{proof}
$\Leftarrow$: Assume for contradiction, that there is a subtour $P\subset V$
with $v(P)\ne V$. Then equation (\ref{eqn:degree}) implies that all $a\in P$
has exactly two edges in the tour incident, and these edges all end in another
$b\in P$. More specifically. No edge involving $a\in P$ can cross the cut
$(S, V\setminus S)$. This is a contradiction to the assumption that all cuts
not seeing the whole graph have at least two edges crossing.

$\Rightarrow$: There are two cases here. Either we have one subtour that can
see the entire case. In this case, it is clear that all cuts
$(S, V\setminus S)$ as described will have at least two edges crossing. If
they didn't the subtour wouldn't be able to see the whole graph, which is a
contradiction. The other case is that we have several subtours of which
at least one can see the entire graph. In this case we would remove all the
others to get a better solution, so we don't care.
\end{proof}

\subsection*{Question 1.4}
The biggest problem with the ILP formulation is that the amount of constraints
is exponential in the input size. Removing \autoref{eqn:subtour} gives a lower
bound for the TCP with a polynomial amount of constraints.

\section*{Part 2}
\subsection*{Question 2.1}

We've alreadydecided to remove \autoref{eqn:subtour} to get a problem of
polynomial size.

This leaves us with a nice set of constraints, however the integer constraints
make it a much harder problem. We therefore relax all constraints
$k\in \{a, b\}$ and replace them with $a\le k \le b$.

\subsection*{Question 2.2}
We implemented this in C++ rather than using the supplied Java code. Our
definition of when a node is ``evaluated'' might therefore be different. We say
that a node is evaluated when it is removed from the priority queue that makes
up the best-first search. This is probably not the same metric as is used in
the java code, so we might have lower numbers.
\\\\
\begin{tabular}{| l | l | l | r |}
\hline
\textbf{Graph} & \textbf{\# eval} & \textbf{Tour} & \textbf{Length} \\
\hline
1              & 44              & 6, 8, 7, 5, 6 & $5.013630$      \\
2              & 4               & 2, 3, 5, 4, 2 & $4.000000$      \\
3              & 9               & 0, 1, 3, 5, 7, 6, 4, 2, 0 & $8.000000$ \\
\hline
\end{tabular}


% That's all folks!
\end{document}
